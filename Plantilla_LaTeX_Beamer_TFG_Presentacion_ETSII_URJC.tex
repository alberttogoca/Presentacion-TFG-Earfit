
%\documentclass[xcolor=svgnames,handout,aspectratio=169]{beamer} % proyector ancho (aspecto 16x9)
\documentclass[xcolor=svgnames,handout]{beamer} % proyector tradicional (aspecto 4x3)

%%%% usa PDFLATEX !!!!!!!!!!!!!!!!!! %%%%



\mode<presentation>
{
  %% THEME %%
  \usetheme{Frankfurt}
  %\usecolortheme{beaver}



\setbeamertemplate{items}[ball]
\setbeamertemplate{blocks}[rounded][shadow=true]
\setbeamertemplate{navigation symbols}{} %%% To get rid of navigating icons


%\usecolortheme{rose}


\usecolortheme[named=Crimson]{structure}



%\setbeamercolor{palette primary}{bg=lightred,fg=darkred}


\setbeamercolor{palette primary}{bg=rojoURJC,fg=white}
\setbeamercolor{palette secondary}{bg=midred,fg=white}
\setbeamercolor{palette tertiary}{bg=lightred,fg=darkred}


\setbeamercolor*{author in head/foot}{parent=palette tertiary}
\setbeamercolor*{title in head/foot}{parent=palette secondary}
\setbeamercolor*{date in head/foot}{parent=palette primary}

% Color de fórmulas. Se puede usar MediumBlue, Crimson, etc.
% En este documento aparecen en negro
\setbeamercolor{math text}{fg=Black}
\setbeamercolor{math text inlined}{parent=math text}
\setbeamercolor{math text displayed}{parent=math text}

\setbeamercolor{alerted text}{fg=Red}


\defbeamertemplate*{footline}{infolines theme}
{
  \leavevmode%
  \hbox{%
  \begin{beamercolorbox}[wd=.333333\paperwidth,ht=2.25ex,dp=1ex,center]{author in head/foot}%
    \usebeamerfont{author in head/foot}\insertshortinstitute
  \end{beamercolorbox}%
  \begin{beamercolorbox}[wd=.333333\paperwidth,ht=2.25ex,dp=1ex,center]{title in head/foot}%
    \usebeamerfont{title in head/foot}\insertshortauthor
  \end{beamercolorbox}%
  \begin{beamercolorbox}[wd=.333333\paperwidth,ht=2.25ex,dp=1ex,right]{date in head/foot}%
    \usebeamerfont{date in head/foot}\insertshortdate{}\hspace*{2em}
    \insertframenumber{} / \inserttotalframenumber\hspace*{2ex}
  \end{beamercolorbox}}%
  \vskip0pt%
}



  %% FONT %%
  %\usefonttheme{default}
  %\usefonttheme{professionalfonts}
  %\usefonttheme{serif}
  \usefonttheme{structurebold}
  %\usefonttheme{structureitalicserif}
  %\usefonttheme{structuresmallcapsserif}


%  \setbeamercovered{transparent}
%  \setbeamercovered{dynamic}
}


\usepackage[spanish]{babel}

%\usepackage[latin1]{inputenc} % Usad en WinEdt/MikTex
\usepackage[utf8]{inputenc} % Usad en overleaf

\usepackage{graphicx, amsmath}

%\usepackage{paralist}

\usepackage{hyperref}
\usepackage{multimedia}
\usepackage{color}


\usepackage{pgfpages}
%\pgfpagesuselayout{2 on 1}[a4paper,border shrink=5mm]
%\pgfpagesuselayout{4 on 1}[landscape,a4paper,border shrink=5mm]

\definecolor{darkblue}{rgb}{0.15,0.15,0.70}
%\definecolor{darkred}{rgb}{0.80,0.20,0.20}


\definecolor{crimson}{rgb}{0.86,0.08,0.24}
\definecolor{darkred}{rgb}{0.65,0.17,0.17}
\definecolor{midred}{rgb}{0.85,0.40,0.40}
\definecolor{lightred}{rgb}{0.95,0.65,0.65}

\definecolor{rojoURJC}{rgb}{0.796,0,0.09}
\definecolor{pureblue}{rgb}{0,0,1}





%%%%%%%%%%%%%%%%%%%%%%% Definiciones básicas %%%%%%%%%%%%%%%%%%%%%%%

\newcommand{\nombreautor}{Alberto Gómez Cano}
\newcommand{\nombretutor}{Manuel Rubio Sánchez}
\newcommand{\titulotrabajo}{EARFIT: Aplicación Para Entrenamiento Auditivo Musical Basada en Next.js y TypeScript}
\newcommand{\escuela}{Escuela Técnica Superior\\de Ingeniería Informática}
\newcommand{\escuelalargo}{Escuela Técnica Superior de Ingeniería Informática}
\newcommand{\universidad}{Universidad Rey Juan Carlos}
\newcommand{\fecha}{\today}
\newcommand{\grado}{Grado en Ingeniería Informática}
\newcommand{\gradoabr}{TFG - GII} % Para el Grado en Ingeniería Informática
\newcommand{\curso}{Curso 2021-2022}
\newcommand{\logoUniversidad}{logoURJC.pdf}
\newcommand{\logoEscuela}{logo_ETSII_estrecho.jpg}

%%%%%%%%%%%%%%%%%%%%%%%%%%%%%%%%%%%%%%%%%%%%%%%%%%%%%%%%%%%%%%%%%%%%


%%%%%%%%%%%%%%%%%%%%%%%%%%%% Comandos definidos por el autor %%%%%%%

\newcommand{\traspuesta}{\mbox{\tiny $\mathsf{T}$}}
\graphicspath{ {assets} }

%%%%%%%%%%%%%%%%%%%%%%%%%%%%%%%%%%%%%%%%%%%%%%%%%%%%%%%%%%%%%%%%%%%%





\title{\titulotrabajo}
\author[\gradoabr]{\footnotesize \textrm{Trabajo fin de grado}}
\institute[\textsc{\nombreautor}]{\large \textbf{\grado} \\ \bigskip   \bigskip   \begin{footnotesize} \textsf{Autor:} \textit{\nombreautor} \\ \smallskip \textsf{Tutor:} \textit{\nombretutor} \end{footnotesize} \\ \bigskip \bigskip \includegraphics[height=1cm]{\logoEscuela}}
\date[\fecha]{%\vspace{-0.7cm}
}





\begin{document}


\begin{frame}[plain]
\maketitle
\end{frame}








% Opcionales
\usebackgroundtemplate{\includegraphics[width=\paperwidth]{logoURJC_marca_agua_U.jpg}}
\logo{\vspace{-0.2cm} \includegraphics[height=0.5cm,clip=true]{icono_grado_informatica.png}}


% ==================== CONTENIDO ========================
\section*{Contenido}

\begin{frame}
  \frametitle{Contenido}
  \tableofcontents
\end{frame}

% ==================== INTRODUCCIÓN ========================
\section{Introducción}
\begin{frame}
  \frametitle{Resumen}
  \begin{itemize}
    \item Herramienta para ayudar a músicos a desarrollar su oído (Musical Ear Training). \bigskip
    \item EARFIT es una PWA basada en Next.js y TypeScript. \bigskip
    \item Desarrollada bajo metodologías ágiles y desplegada en Vercel. \bigskip
    \item Se basa en un conjunto de ejercicios de entrenamiento auditivo. \bigskip
  \end{itemize}
\end{frame}

\begin{frame}
  \frametitle{Entrenamiento Auditivo}
\begin{itemize}
  \item Es el proceso de identificar y asociar los elementos musicales con la forma en que se percibe el sonido. \bigskip
  \item Los músicos, productores y DJs pueden beneficiarse del entrenamiento auditivo. \bigskip
  \item Permite sacar canciones más rápido, con mayor precisión, improvisar mejor y llevar al instrumento las melodías que imagines con mayor facilidad. \bigskip
  \item Los ejercicios más comunes incluyen habilidades como identificar notas, intervalos, escalas... \bigskip
\end{itemize}
\end{frame}

% ==================== OBJETIVOS ========================
\begin{frame}
  \frametitle{Objetivos}

  \begin{itemize}
    \item \textbf{Objetivo principal}: \bigskip
    \begin{itemize}
      \item Crear una aplicación que permita a músicos desarrollar su oído musical mediante el entrenamiento auditivo. \bigskip
    \end{itemize}
    \item \textbf{Subobjetivos}: \bigskip
    \begin{itemize}
      \item Desarrollar una interfaz interactiva. \bigskip
      \item Implementar diferentes tipos de ejercicio personalizables. \bigskip
      \item Incluir varios instrumentos para prácticar con sus sonidos. \bigskip
    \end{itemize}
  \end{itemize}

\end{frame}

% ==================== METODOLOGIAS ========================
\section{Metodologías}
\begin{frame}
  \frametitle{Metodologías}
  \framesubtitle{Proceso Combinado de Design Thinking, Lean Startup, Scrum y DevOps}
  \centering
  \includegraphics[clip=true,width=\textwidth]{LeanDesignAgile}\\
\end{frame}

\subsection{Design Thinking}
\begin{frame}
  \frametitle{Design Thinking}
  \framesubtitle{Generar Ideas Innovadoras}
  \centering
  \includegraphics[clip=true,width=\textwidth]{Design Thinking/Etapas}\\
\end{frame}
\begin{frame}
  \frametitle{Mindmap (1/2)}
  \framesubtitle{Representar Ideas o Conceptos y Encontrar Soluciones}
  \centering
  \includegraphics[clip=true,width=0.65\textwidth]{Design Thinking/MindMap}\\
\end{frame}
\begin{frame}
  \frametitle{Mindmap (2/2)}
  \framesubtitle{User Persona, Problemas y Soluciones (Hipótesis)}
  \centering
  \includegraphics[clip=true,width=\textwidth]{Design Thinking/MindMap2}\\
\end{frame}
\begin{frame}
  \frametitle{MoSCoW (1/2)}
  \framesubtitle{Establecer las Prioridades del Proyecto}
  \centering
  \includegraphics[clip=true,width=0.73\textwidth]{Design Thinking/MosCow}\\
\end{frame}
\begin{frame}
  \frametitle{MoSCoW (2/2)}
  \framesubtitle{Must Have y Should Have}
  \centering
  \includegraphics[clip=true,width=\textwidth]{Design Thinking/MosCow2}\\
\end{frame}
\begin{frame}
  \frametitle{Prototipo (1/2)}
  \framesubtitle{Mobile First y Atomic Design}
  \centering
  \includegraphics[clip=true,width=\textwidth]{Design Thinking/Prototipo/Prototipos}\\
\end{frame}
\begin{frame}
  \frametitle{Prototipo (2/2)}
  \framesubtitle{Pantallas Pequeñas}
  \centering
  \includegraphics[clip=true,width=\textwidth]{Design Thinking/Prototipo/Small/Prototipo1}\\
\end{frame}

\subsection{Lean Startup}
\begin{frame}
  \frametitle{Lean Startup}
  \framesubtitle{Puesta en Marcha y Optimización de la Solución}
  \centering
  \includegraphics[clip=true,width=0.8\textwidth]{Lean Startup/CircuitoFeedback}\\
\end{frame}

\subsection{Scrum}
\begin{frame}
  \frametitle{Scrum}
  \framesubtitle{Proceso de Gestión del Desarrollo}
  \centering
  \includegraphics[clip=true,width=\textwidth]{Scrum/Scrum}\\
\end{frame}
\begin{frame}
  \frametitle{User Story Map (1/2)}
  \framesubtitle{Definir el Viaje o Casos de Uso del Usuario en el Producto}
  \centering
  \includegraphics[clip=true,width=\textwidth]{Scrum/UserStoryMap}\\
\end{frame}
\begin{frame}
  \frametitle{User Story Map (2/2)}
  \framesubtitle{User Activities, User Tasks, User Stories y Releases}
  \centering
  \includegraphics[clip=true,width=\textwidth]{Scrum/UserStoryMap11}\\
\end{frame}
\begin{frame}
  \frametitle{Scrum Board}
  \framesubtitle{Visualizar el Trabajo y Gestionar el Desarrollo / Product Backlog y Sprint Backlog}
  \centering
  \includegraphics[clip=true,width=\textwidth]{Scrum/ScrumBoard1}\\
\end{frame}
\begin{frame}
  \frametitle{User Stories}
  \framesubtitle{Características o Requisitos del Sistema desde la Perspectiva del Usuario}
  \centering
  \includegraphics[clip=true,width=\textwidth]{Scrum/UserStory-small}\\
\end{frame}
% \begin{frame}
%   \frametitle{User Stories (2/2)}
%   \framesubtitle{Continuación...}
%   \centering
%   \includegraphics[clip=true,width=\textwidth]{Scrum/UserStory2}\\
% \end{frame}

\subsection{DevOps}
\begin{frame}
  \frametitle{DevOps}
  \framesubtitle{Agilizar los Procesos del Entorno de Desarrollo al de Producción}
  \centering
  \includegraphics[clip=true,width=\textwidth]{DevOps/DevOps}\\
\end{frame}
\begin{frame}
  \frametitle{Integración Continua (CI)}
  \framesubtitle{Git, GitHub y GitFlow}
  \centering
  \includegraphics[clip=true,width=\textwidth]{DevOps/GitFlow-contrast}\\
\end{frame}
\begin{frame}
  \frametitle{Despliegue Continuo (CD)}
  \framesubtitle{GitHub Actions, Vercel y Flujo DPS}
  \centering
  \includegraphics[clip=true,width=0.8\textwidth]{DevOps/DPS}\\
\end{frame}

% ==================== IMPLEMENTACIÓN ========================
\section{Desarrollo}
\subsection{Tecnologías}
% \begin{frame}
%   \frametitle{Stack Tecnológico}
%   \centering
%   \includegraphics[clip=true,width=\textwidth]{Logos/Logos}\\
% \end{frame}
\begin{frame}
  \frametitle{Stack Tecnológico}
  \centering
  \includegraphics[clip=true,width=0.8\textwidth]{Logos/Logos-small}\\ \bigskip
  \begin{itemize}
    \item \textbf{Next.js}: Enrutamiento basado en páginas, Prerendering, Code Splitting, Prefetching, Fast Refrest, SWC y WebPack... \bigskip
    \item \textbf{React}: Componentes, DOM Virtual, Hooks, Context... \bigskip
    % \item \textbf{Vercel}: Vercel Edge Network, Vercel Analytics, Flujo DPS... \bigskip
    \item \textbf{Typescript}: Tipos estáticos para JavaScript.  \bigskip
    \item \textbf{Node.js}: Entorno de ejecución, librerías (NPM vs. Yarn). \bigskip
    \item \textbf{VScode}:  Git integrado, extensiones (ESLint y Prettier). \bigskip
  \end{itemize}
\end{frame}

\subsection{Detalles de Implementación}
\begin{frame}
  \frametitle{Estructura de Archivos} 
  \framesubtitle{Pages y Public son Directorios Especiales en Next.js}
  \centering
  \includegraphics[clip=true,width=\textwidth]{Detalles de Implementación/Estructura de Archivos/ArchivosConfig}\\
\end{frame}

\begin{frame}
  \frametitle{Arquitectura}
  \framesubtitle{Estructura, Funcionamiento e Interacción}
  \centering
  \includegraphics[clip=true,width=0.85\textwidth]{Detalles de Implementación/Arquitectura2}\\
\end{frame}

\begin{frame}
  \frametitle{Jerarquía de Componentes}
  \framesubtitle{Los Componentes Son Reutilizados entre Páginas}
  \centering
  \includegraphics[clip=true,width=\textwidth]{Detalles de Implementación/JerarquíaComponentes}\\
\end{frame}

\begin{frame}
  \frametitle{Hooks}
  \framesubtitle{Los Custom Hooks Encapsulan la Lógica de Estado}
  \begin{itemize}
    \item \textbf{useExercise.tsx}: Establecer las respuestas para el ejercicio. \smallskip
    \item \textbf{useAnswerToggles.tsx}: Añadir y quitar respuestas a la pregunta. \smallskip
    \item \textbf{useAnswerButtons.tsx}: La lógica de los botones de respuesta. \smallskip
    \item \textbf{useAnswer.tsx}: Calcular la respuesta a preguntar. \smallskip
    \item \textbf{usePlayButton.tsx}: Reproducir la respuesta. \bigskip
    \item \textbf{EarfitContext.tsx}: Establecer y seleccionar los instrumentos. Se usa con el Hook useInstrumentContext().\bigskip
  \end{itemize}
\end{frame}

\begin{frame}
  \frametitle{Servicios}
  \framesubtitle{Proveen los Datos a los Hooks y al Context}
  \begin{itemize}
    \item \textbf{instrumentService.ts}: Provee los instrumentos de la aplicación. \bigskip
    \item \textbf{noteService.ts}: Provee las respuestas correspondientes al ejericio de notas. \bigskip
    \item \textbf{intervalService.ts}: Provee las respuestas correspondientes al ejericio de intervalos. \bigskip
    \item \textbf{scaleService.ts}: Provee las respuestas correspondientes al ejericio de escalas. \bigskip
  \end{itemize}
\end{frame}

\begin{frame}
  \frametitle{Comportamiento de una Página}
  \framesubtitle{Interacción entre Componentes, Hooks y Servicios}
  \includegraphics[clip=true,width=\textwidth]{Detalles de Implementación/Arquitectura1}\\
\end{frame}

\begin{frame}
  \frametitle{Librerías}
  \framesubtitle{Soundfont-wrapper crea los Noteplayer para Cada Instrumento}
  \begin{itemize}
    \item \textbf{Tonal.js}: Manipular elementos musicales. \bigskip
    \item \textbf{Soundfont-player}: Cargar y reproducir archivos MIDI.js. \bigskip
    \item \textbf{Soundfont-wrapper}: Refinar la complejidad de “soundfont-player” y simplificar su uso. \bigskip
    \item \textbf{React-piano}: Teclado de piano interactivo (sin sonidos). \bigskip
    \item \textbf{React-use-measure}: Para que el piano sea responsive. \bigskip
    \item \textbf{React-bootstrap}: Librería de estilos CSS. \bigskip
    \item \textbf{Next-pwa}: Registrar y generar un Service Worker. \bigskip
  \end{itemize}
\end{frame}

\begin{frame}
  \frametitle{Tipos de TypeScript}
  \framesubtitle{Los Instrumentos y las Respuestas}
  \centering
  \includegraphics[clip=true,width=\textwidth]{Detalles de Implementación/Code/Types/Instrument}\\
  \includegraphics[clip=true,width=\textwidth]{Detalles de Implementación/Code/Types/Answer}\\
\end{frame}

% \begin{frame}
%   \frametitle{Tipos de TypeScript (1/2)}
%   \framesubtitle{Cada Instrumento tiene un NotePlayer para Reproducir Notas}
%   \centering
%   \includegraphics[clip=true,width=\textwidth]{Detalles de Implementación/Code/Types/Instrument}\\
%   \includegraphics[clip=true,width=\textwidth]{Detalles de Implementación/Code/Types/Noteplayer-small}\\
% \end{frame}

% \begin{frame}
%   \frametitle{Tipos de TypeScript (2/2)}
%   \framesubtitle{Dependiendo de la Variante, las Respuestas tienen Notas, Intervalos o Escalas}
%   \includegraphics[clip=true,width=\textwidth]{Detalles de Implementación/Code/Types/VariantExercise}\\
%   \includegraphics[clip=true,width=\textwidth]{Detalles de Implementación/Code/Types/Answer}\\
% \end{frame}

\begin{frame}
  \frametitle{Ejemplo (1/3)}
  \framesubtitle{Llamada a los Hooks Necesarios}
  \includegraphics[clip=true,width=\textwidth]{Detalles de Implementación/Code/Pages/Page1}\\
\end{frame}

\begin{frame}
  \frametitle{Ejemplo (2/3)}
  \framesubtitle{Los Componentes reciben Estados y Callbacks a Través de Props}
  \includegraphics[clip=true,width=\textwidth]{Detalles de Implementación/Code/Pages/Page2}\\
\end{frame}

\begin{frame}
  \frametitle{Ejemplo (3/3)}
  \framesubtitle{Los Componentes Actúan como Funciones Puras}
  \includegraphics[clip=true,width=\textwidth]{Detalles de Implementación/Code/Components/Component}\\
\end{frame}

\subsection{Progressive Web App}
\begin{frame}
  \frametitle{Progressive Web App (1/3)}
  \framesubtitle{Confiable e Instalable como una App Nativa en PC, Móvil y Tablet}
  Para que una aplicación sea PWA debe tener: \bigskip
  \begin{itemize}
    \item Una conexión segura HTTPS. \bigskip
    \item Cargue sin conexión, para ello requiere un Service Worker. \bigskip
    \item Información como nombre, autor, icono y descripción en un documento JSON llamado Manifest. \bigskip
    \item Un icono de al menos 144x144 px en formato PNG. \
  \end{itemize}
\end{frame}
\begin{frame}
  \frametitle{Progressive Web App (2/3)}
  \framesubtitle{¿Cómo se Instala?}
  En Safari, la opción se llama ``añadir a pantalla de inicio'' y en Chrome aparece un icono en la barra de búsqueda. \bigskip

  \centering
  \includegraphics[clip=true,width=0.5\textwidth]{Detalles de Implementación/PWA/InstalarPWA}\\
\end{frame}
\begin{frame}
  \frametitle{Progressive Web App (3/3)}
  \framesubtitle{Earfit como PWA en MacOs}
  \centering
  \includegraphics[clip=true,width=\textwidth]{Capturas Earfit/PC/Intervals}\\
\end{frame}

\subsection{Software QA}
\begin{frame}
  \frametitle{Google Lighthouse}
  \framesubtitle{Auditoría de Calidad de la Página Web}
  \centering
  \includegraphics[clip=true,width=\textwidth]{Lighthouse/Resumen}\\
  \raggedright	
  En el apartado Performance: \bigskip
  \begin{itemize}
    \item Lighthouse da falsas mediciones para aplicaciones Next.js. \bigskip
    \item Lighthouse estima las Web Vitals ejecutando una simulación. \bigskip
    \item En este caso, usar Vercel Analytics aporta ventajas, como datos reales de los dispositivos de los usuarios. \bigskip
  \end{itemize}

\end{frame}
\begin{frame}
  \frametitle{Vercel Analytics}
  \framesubtitle{Experiencia de Usuario de la Página Web (Web Vitals)}
  \centering
  \includegraphics[clip=true,width=\textwidth]{Vercel/VercelAnalytics}\\
\end{frame}


% ==================== CONCLUSIONES ========================
\section{Conclusiones}
\begin{frame}
  \frametitle{Conclusiones}
  \textbf{Objetivos Alcanzados}: \bigskip
    \begin{itemize}
      \item PWA que permite desarrollar el oído musical mediante entrenamiento auditivo. \bigskip
      \item Diferentes tipos de ejercicios personalizables. \bigskip
      \item Varios instrumentos para practicar con sus sonidos. \bigskip
      \item Buenas prácticas, gestión ágil y DevOps. \bigskip
    \end{itemize}
    \textbf{Trabajos Futuros}: \bigskip
    \begin{itemize}
      \item Ejercicios de acordes, ritmos, progresiones, modo nocturno y varios idiomas. \bigskip
      % \item Añadir cobertura de código y test automáticos al flujo de Integración Continua (CI) \smallskip
      % \item Considerar Software libre vs Negocio: Plantear alguna forma de monetización, marketing, etc. \bigskip
    \end{itemize}
\end{frame}

% ==================== DEMO ========================
\begin{frame}
  \frametitle{Demostración}
  \centering
  \href{https://earfit.vercel.app/}{\includegraphics[width=0.6\textheight, keepaspectratio]{Logos/Earfit2}}
 % \href{https://github.com/alberttogoca/EarFit}{Código en GitHub} \bigskip
 % \includegraphics[clip=true,width=0.1\textwidth]{Conclusiones/EarfitQR} \bigskip
\end{frame}

% \begin{frame}{Video on the computer}
%     \centering
%     \movie[externalviewer]{\includegraphics[width=\textheight, keepaspectratio]{figures/image.jpg}}{video_beamer.mp4}
% \end{frame}

% ==================== ÚLTIMA DIAPOSITIVA ========================

\setbeamercolor{math text}{fg=Crimson}
\setbeamercolor{math text inlined}{parent=math text}
\setbeamercolor{math text displayed}{parent=math text}

\usebackgroundtemplate{\includegraphics[width=\paperwidth]{fondo_blanco.jpg}}

\begin{frame}[plain]

  \begin{center}

    \bigskip  \bigskip

    \begin{Large} \textcolor{crimson}{\textbf{\titulotrabajo}} \end{Large} \bigskip \bigskip

    \begin{footnotesize} \textrm{Trabajo Fin de Grado} \\ \end{footnotesize} \bigskip \bigskip

    \begin{footnotesize} \textbf{\grado} -- \curso \end{footnotesize} \bigskip \bigskip

    \begin{footnotesize} \textsf{Autor:} \textit{\nombreautor} \\ \smallskip
                          \textsf{Tutor:} \textit{\nombretutor} \end{footnotesize} \bigskip \bigskip


    \includegraphics[height=1cm]{logo_ETSII_estrecho.jpg}

  \end{center}

\end{frame}

\end{document}


% ==================== Hooks y Context ========================

  % \subsection{Hooks y Context}
% \begin{frame}
%   \frametitle{Hooks}
%   \begin{itemize}
%     \item \textbf{useExercise.tsx} \smallskip
%     \item \textbf{useAnswerToggles.tsx} \smallskip
%     \item \textbf{useAnswer.tsx} \smallskip
%     \item \textbf{useAnswerButtons.tsx} \smallskip
%     \item \textbf{useStreak.tsx} \smallskip
%     \item \textbf{usePlayButton.tsx} \smallskip
%     \item \textbf{useScaleDropDown.tsx} \smallskip
%     \item \textbf{usePiano.tsx} \bigskip
%   \end{itemize}
% \end{frame}

% \begin{frame}
%   \frametitle{Context}
%   \begin{itemize}
%     \item \textbf{EarfitContex.tsx}: Se encarga de la lógica relacionada a los instrumentos.\bigskip
%   \end{itemize}
% \end{frame}

% ==================== Tipos de TypeScript ========================

% \begin{frame}
%   \frametitle{Tipos de TypeScript (1/2)}
%   \framesubtitle{Cada Instrumento tiene un NotePlayer para Reproducir Notas}
%   \centering
%   \includegraphics[clip=true,width=\textwidth]{Detalles de Implementación/Code/Types/Instrument}\\
%   \includegraphics[clip=true,width=\textwidth]{Detalles de Implementación/Code/Types/Noteplayer-small}\\
% \end{frame}

% \begin{frame}
%   \frametitle{Tipos de TypeScript (2/2)}
%   \framesubtitle{Dependiendo de la Variante, las Respuestas tienen Notas, Intervalos o Escalas}
%   \includegraphics[clip=true,width=\textwidth]{Detalles de Implementación/Code/Types/VariantExercise}\\
%   \includegraphics[clip=true,width=\textwidth]{Detalles de Implementación/Code/Types/Answer}\\
% \end{frame}


% % ==================== Ejemplos Latex========================
% \begin{frame}
%   \frametitle{Ejemplo con colores}

%   \begin{itemize}
%     \item $\mathbf{x}$: Data sample \smallskip
%     \item $\mathbf{p}$: Low-dimensional representation of $\mathbf{x}$  \smallskip
%     \item $\hat{\mathbf{x}} = \mathbf{V}\mathbf{p}$: Estimates of $\mathbf{x}$ \smallskip
%     \item $\hat{x}_{i} = \mathbf{v}_{i}^{\traspuesta}\mathbf{p}$: Estimate of $\mathbf{x}$ for the $i$-th attribute \smallskip
%     \begin{itemize}
%       \item Through orthogonal projections onto $i$-th axis
%     \end{itemize}
%   \end{itemize}

% \end{frame}

%  % === Ejemplo con minipage ===
%  \usebackgroundtemplate{\includegraphics[width=\paperwidth]{fondo_blanco.jpg}}
%  \begin{frame}
%    \frametitle{Ejemplo con minipage}
 
%    \begin{minipage}{0.45\textwidth}
%     \centering
 
%  \begin{itemize}
%    \item Teorema de Pitágoras \medskip
%    \item Hipotenusa  \medskip
%    \item Catetos \medskip
%    \item Triángulo rectángulo \medskip
%    \begin{itemize}
%      \item Incluid espacios verticales \smallskip
%      \item Se ha quitado la marca de agua
%    \end{itemize}
%  \end{itemize}
 
%    \end{minipage}
%    \hfill
%    \begin{minipage}{0.45\textwidth}
%     \centering
%       \includegraphics[clip=true,width=\textwidth]{triangulos_separados_bb.pdf}\\
%    \end{minipage}
 
%  \end{frame}
 
%  % Restablecer la marca de agua
%  \usebackgroundtemplate{\includegraphics[width=\paperwidth]{logoURJC_marca_agua_U.jpg}}
%  % === Fin Ejemplo con minipage ===
