
%\documentclass[xcolor=svgnames,handout,aspectratio=169]{beamer} % proyector ancho (aspecto 16x9)
\documentclass[xcolor=svgnames,handout]{beamer} % proyector tradicional (aspecto 4x3)

%%%% usa PDFLATEX !!!!!!!!!!!!!!!!!! %%%%



\mode<presentation>
{
  %% THEME %%
  \usetheme{Frankfurt}
  %\usecolortheme{beaver}



\setbeamertemplate{items}[ball]
\setbeamertemplate{blocks}[rounded][shadow=true]
\setbeamertemplate{navigation symbols}{} %%% To get rid of navigating icons


%\usecolortheme{rose}


\usecolortheme[named=Crimson]{structure}



%\setbeamercolor{palette primary}{bg=lightred,fg=darkred}


\setbeamercolor{palette primary}{bg=rojoURJC,fg=white}
\setbeamercolor{palette secondary}{bg=midred,fg=white}
\setbeamercolor{palette tertiary}{bg=lightred,fg=darkred}


\setbeamercolor*{author in head/foot}{parent=palette tertiary}
\setbeamercolor*{title in head/foot}{parent=palette secondary}
\setbeamercolor*{date in head/foot}{parent=palette primary}

% Color de fórmulas. Se puede usar MediumBlue, Crimson, etc.
% En este documento aparecen en negro
\setbeamercolor{math text}{fg=Black}
\setbeamercolor{math text inlined}{parent=math text}
\setbeamercolor{math text displayed}{parent=math text}

\setbeamercolor{alerted text}{fg=Red}


\defbeamertemplate*{footline}{infolines theme}
{
  \leavevmode%
  \hbox{%
  \begin{beamercolorbox}[wd=.333333\paperwidth,ht=2.25ex,dp=1ex,center]{author in head/foot}%
    \usebeamerfont{author in head/foot}\insertshortinstitute
  \end{beamercolorbox}%
  \begin{beamercolorbox}[wd=.333333\paperwidth,ht=2.25ex,dp=1ex,center]{title in head/foot}%
    \usebeamerfont{title in head/foot}\insertshortauthor
  \end{beamercolorbox}%
  \begin{beamercolorbox}[wd=.333333\paperwidth,ht=2.25ex,dp=1ex,right]{date in head/foot}%
    \usebeamerfont{date in head/foot}\insertshortdate{}\hspace*{2em}
    \insertframenumber{} / \inserttotalframenumber\hspace*{2ex}
  \end{beamercolorbox}}%
  \vskip0pt%
}



  %% FONT %%
  %\usefonttheme{default}
  %\usefonttheme{professionalfonts}
  %\usefonttheme{serif}
  \usefonttheme{structurebold}
  %\usefonttheme{structureitalicserif}
  %\usefonttheme{structuresmallcapsserif}


%  \setbeamercovered{transparent}
%  \setbeamercovered{dynamic}
}


\usepackage[spanish]{babel}

%\usepackage[latin1]{inputenc} % Usad en WinEdt/MikTex
\usepackage[utf8]{inputenc} % Usad en overleaf

\usepackage{graphicx, amsmath}

%\usepackage{paralist}

\usepackage{hyperref}
\usepackage{multimedia}
\usepackage{color}


\usepackage{pgfpages}
%\pgfpagesuselayout{2 on 1}[a4paper,border shrink=5mm]
%\pgfpagesuselayout{4 on 1}[landscape,a4paper,border shrink=5mm]

\definecolor{darkblue}{rgb}{0.15,0.15,0.70}
%\definecolor{darkred}{rgb}{0.80,0.20,0.20}


\definecolor{crimson}{rgb}{0.86,0.08,0.24}
\definecolor{darkred}{rgb}{0.65,0.17,0.17}
\definecolor{midred}{rgb}{0.85,0.40,0.40}
\definecolor{lightred}{rgb}{0.95,0.65,0.65}

\definecolor{rojoURJC}{rgb}{0.796,0,0.09}
\definecolor{pureblue}{rgb}{0,0,1}





%%%%%%%%%%%%%%%%%%%%%%% Definiciones básicas %%%%%%%%%%%%%%%%%%%%%%%

\newcommand{\nombreautor}{Alberto Gómez Cano}
\newcommand{\nombretutor}{Manuel Rubio Sánchez}
\newcommand{\titulotrabajo}{EARFIT: Aplicación Para Entrenamiento Auditivo Musical Basada en Next.js y TypeScript}
\newcommand{\escuela}{Escuela Técnica Superior\\de Ingeniería Informática}
\newcommand{\escuelalargo}{Escuela Técnica Superior de Ingeniería Informática}
\newcommand{\universidad}{Universidad Rey Juan Carlos}
\newcommand{\fecha}{\today}
\newcommand{\grado}{Grado en Ingeniería Informática}
\newcommand{\gradoabr}{TFG - GII} % Para el Grado en Ingeniería Informática
\newcommand{\curso}{Curso 2021-2022}
\newcommand{\logoUniversidad}{logoURJC.pdf}
\newcommand{\logoEscuela}{logo_ETSII_estrecho.jpg}

%%%%%%%%%%%%%%%%%%%%%%%%%%%%%%%%%%%%%%%%%%%%%%%%%%%%%%%%%%%%%%%%%%%%


%%%%%%%%%%%%%%%%%%%%%%%%%%%% Comandos definidos por el autor %%%%%%%

\newcommand{\traspuesta}{\mbox{\tiny $\mathsf{T}$}}
\graphicspath{ {assets} }

%%%%%%%%%%%%%%%%%%%%%%%%%%%%%%%%%%%%%%%%%%%%%%%%%%%%%%%%%%%%%%%%%%%%





\title{\titulotrabajo}
\author[\gradoabr]{\footnotesize \textrm{Trabajo fin de grado}}
\institute[\textsc{\nombreautor}]{\large \textbf{\grado} \\ \bigskip   \bigskip   \begin{footnotesize} \textsf{Autor:} \textit{\nombreautor} \\ \smallskip \textsf{Tutor:} \textit{\nombretutor} \end{footnotesize} \\ \bigskip \bigskip \includegraphics[height=1cm]{\logoEscuela}}
\date[\fecha]{%\vspace{-0.7cm}
}





\begin{document}


\begin{frame}[plain]
\maketitle
\end{frame}








% Opcionales
\usebackgroundtemplate{\includegraphics[width=\paperwidth]{logoURJC_marca_agua_U.jpg}}
\logo{\vspace{-0.2cm} \includegraphics[height=0.5cm,clip=true]{icono_grado_informatica.png}}


% ==================== CONTENIDO ========================
\section*{Contenido}

\begin{frame}
  \frametitle{Contenido}
  \tableofcontents
\end{frame}

% ==================== INTRODUCCIÓN ========================
\section{Introducción}

\begin{frame}
  \frametitle{Entrenamiento Auditivo}

\end{frame}

% ==================== OBJETIVOS ========================
\section{Objetivos}

\begin{frame}
  \frametitle{Objetivos del TFG}

  \begin{itemize}
    \item Objetivo principal \smallskip
    \begin{itemize}
      \item Desarrollar una aplicación que permita a músicos desarrollar su oído musical mediante el entrenamiento auditivo. \bigskip
    \end{itemize}
    \item Subobjetivos \smallskip
    \begin{itemize}
      \item Desarrollar una interfaz interactiva. \smallskip
      \item Implementar diferentes tipos de ejercicio personalizables de entrenamiento auditivo. \smallskip
      \item Incluir varios instrumentos para prácticar con diferentes sonidos. \bigskip
    \end{itemize}
  \end{itemize}

\end{frame}

% ==================== METODOLOGIAS ========================
\section{Metodologías}
\begin{frame}
  \frametitle{Metodologías}
  \centering
  \includegraphics[clip=true,width=\textwidth]{Design Thinking/LeanDesignAgile}\\
\end{frame}

\subsection{Design Thinking}
\begin{frame}
  \frametitle{Design Thinking}
  \framesubtitle{Generar Ideas Innovadoras}
  \centering
  \includegraphics[clip=true,width=\textwidth]{Design Thinking/Etapas}\\
\end{frame}
\begin{frame}
  \frametitle{Mindmap}
  \framesubtitle{Etapa: Idear}
  \centering
  \includegraphics[clip=true,width=0.64\textwidth]{Design Thinking/MindMap}\\
\end{frame}
\begin{frame}
  \frametitle{MoSCoW}
  \framesubtitle{Etapa: Protoptipar}
  \centering
  \includegraphics[clip=true,width=0.7\textwidth]{Design Thinking/MosCow}\\
\end{frame}
\begin{frame}
  \frametitle{Protoptipo}
  \framesubtitle{Etapa: Protoptipar}
  \centering
  \includegraphics[clip=true,width=0.9\textwidth]{Design Thinking/Prototipo/Small/Prototipo1}\\
\end{frame}

\subsection{Lean Startup}
\begin{frame}
  \frametitle{Lean Startup}
  \framesubtitle{Agilizar la Puesta en Marcha}
  \centering
  \includegraphics[clip=true,width=0.8\textwidth]{Lean Startup/CircuitoFeedback}\\
\end{frame}

\subsection{Scrum}
\begin{frame}
  \frametitle{Scrum}
  \framesubtitle{Proceso de Gestión}
  \centering
  \includegraphics[clip=true,width=\textwidth]{Scrum/Scrum}\\
\end{frame}
\begin{frame}
  \frametitle{User Story Map}
  \centering
  \includegraphics[clip=true,width=\textwidth]{Scrum/UserStoryMap}\\
\end{frame}
\begin{frame}
  \frametitle{Scrum Board}
  \centering
  \includegraphics[clip=true,width=\textwidth]{Scrum/ScrumBoard}\\
\end{frame}
\begin{frame}
  \frametitle{User Stories}
  \centering
  \includegraphics[clip=true,width=0.6\textwidth]{Scrum/UserStory}\\
\end{frame}

\subsection{DevOps}
\begin{frame}
  \frametitle{DevOps}
  \framesubtitle{Filosofía de Desarrollo}
  \centering
  \includegraphics[clip=true,width=\textwidth]{DevOps/DevOps}\\
\end{frame}
\begin{frame}
  \frametitle{Integración Continua (CI)}
  \centering
  \includegraphics[clip=true,width=\textwidth]{DevOps/GitFlow}\\
\end{frame}
\begin{frame}
  \frametitle{Despliegue Continuo (CD)}
  \centering
  \includegraphics[clip=true,width=0.8\textwidth]{DevOps/DPS}\\
\end{frame}

% ==================== IMPLEMENTACIÓN ========================
\section{Desarrollo del Software}
\subsection{Tecnologías}
\begin{frame}
  \frametitle{Stack Tecnológico}
  \begin{itemize}
    \item \textbf{Next.js}: Enrutamiento basado en páginas, Prerendering, Code Splitting, Prefetching, Fast Refrest, SWC y WebPack... \smallskip
        \pause
    \item \textbf{React}: Componentes reutilizables, DOM Virtual, Hooks, Context... \smallskip
        \pause
    \item \textbf{Vercel}: Vercel Edge Network, Vercel Analytics, Flujo DPS... \smallskip
        \pause
    \item \textbf{Typescript}: Tipos estáticos para JavaScript.  \smallskip
        \pause
    \item \textbf{Node.js}: Entorno de ejecución Javascript, librerías útiles (NPM vs. Yarn). \smallskip
        \pause
    \item \textbf{VScode}:  Git integrado, depuración, resaltado de sintaxis, finalización inteligente, refactorización de código, extensiones (ESLint y Prettier)... \bigskip
  \end{itemize}
\end{frame}

\subsection{Arquitectura}
\begin{frame}
  \frametitle{Arquitectura}
  \centering
  \includegraphics[clip=true,width=0.85\textwidth]{Detalles de Implementación/Arquitectura}\\
\end{frame}

\subsection{Estructura de Archivos}
\begin{frame}
  \frametitle{Estructura de Archivos}
  \centering
  \includegraphics[clip=true,width=\textwidth]{Detalles de Implementación/Estructura de Archivos/ArchivosConfig}\\
\end{frame}

\subsection{Detalles de Implementación}
\begin{frame}
  \frametitle{Tipos}
  \centering
  \includegraphics[clip=true,width=\textwidth]{Detalles de Implementación/Code/Types/Instrument}\\
  \includegraphics[clip=true,width=\textwidth]{Detalles de Implementación/Code/Types/Noteplayer-small}\\
\end{frame}

\begin{frame}
  \includegraphics[clip=true,width=\textwidth]{Detalles de Implementación/Code/Types/VariantExercise}\\
  \includegraphics[clip=true,width=\textwidth]{Detalles de Implementación/Code/Types/Answer}\\
\end{frame}

\begin{frame}
  \frametitle{Librerías}
  \begin{itemize}
    \item \textbf{Tonal.js}: Manipular elementos musicales. \smallskip
    \item \textbf{Soundfont-player}: Cargar y reproducir archivos MIDI.js. \smallskip
    \item \textbf{Soundfont-wrapper}: Refinar la complejidad de “soundfont-player” y simplificar su uso. \smallskip
    \item \textbf{React-piano}: Teclado de piano interactivo (sin sonidos). \smallskip
    \item \textbf{React-use-measure}: Para que el piano sea responsive. \smallskip
    \item \textbf{React-bootstrap}: Librería de estilos CSS. \smallskip
    \item \textbf{Next-pwa}: Registrar y generar un Service Worker. \bigskip
  \end{itemize}
\end{frame}

\begin{frame}
  \frametitle{Servicios}
  \begin{itemize}
    \item \textbf{instrumentService.ts}: Provee los instrumentos de la aplicación. \smallskip
    \item \textbf{noteService.ts}: Provee las respuestas correspondientes al ejericio de notas. \smallskip
    \item \textbf{intervalService.ts}: Provee las respuestas correspondientes al ejericio de intervalos. \smallskip
    \item \textbf{scaleService.ts}: Provee las respuestas correspondientes al ejericio de escalas. \bigskip
  \end{itemize}
\end{frame}

\begin{frame}
  \frametitle{Jerarquía de Componentes}
  \includegraphics[clip=true,width=\textwidth]{Detalles de Implementación/JerarquíaComponentes}\\
\end{frame}

\begin{frame}
  \frametitle{Comportamiento de una Página (1/2)}
  \framesubtitle{Hooks}
  \includegraphics[clip=true,width=\textwidth]{Detalles de Implementación/Code/Pages/Page1}\\
\end{frame}

\begin{frame}
  \frametitle{Comportamiento de una Página (2/2)}
  \framesubtitle{Componentes}
  \includegraphics[clip=true,width=\textwidth]{Detalles de Implementación/Code/Pages/Page2}\\
\end{frame}

\begin{frame}
  \frametitle{Comportamiento de un Componente}
  \includegraphics[clip=true,width=\textwidth]{Detalles de Implementación/Code/Components/Component}\\
\end{frame}

\subsection{Progressive Web App}
\begin{frame}
  \frametitle{Progressive Web App (PWA)}
  Una PWA es confiable, instalable y se comporta como una app nativa en ordenador, móvil y tablet. \bigskip

  Para que una aplicación sea PWA debe tener: 
  \begin{itemize}
    \item Una conexión segura HTTPS. \smallskip
    \item Cargue sin conexión, para ello requiere un Service Worker. \smallskip
    \item Información como nombre, autor, icono y descripción en un documento JSON llamado Manifest. \smallskip
    \item Un icono de al menos 144x144 px en formato PNG. \bigskip
  \end{itemize}
\end{frame}
\begin{frame}
  \frametitle{Progressive Web App (PWA)}
  \framesubtitle{Instalar}
  En iOS, la opción se llama “añadir a pantalla de inicio” desde Safari y en Chrome aparece un icono en la barra de búsqueda.
  \centering
  \includegraphics[clip=true,width=0.5\textwidth]{Detalles de Implementación/PWA/InstalarPWA}\\
\end{frame}
\begin{frame}
  \frametitle{Progressive Web App (PWA)}
  \framesubtitle{Earfit como PWA en MacOs}
  \centering
  \includegraphics[clip=true,width=\textwidth]{Capturas Earfit/PC/Intervals}\\
\end{frame}

\subsection{Software QA}
\begin{frame}
  \frametitle{Google Lighthouse}
  \framesubtitle{Auditoría de Calidad de la Página Web}
  \centering
  \includegraphics[clip=true,width=\textwidth]{Lighthouse/Resumen}\\
  \raggedright	
  En el apartado Performance:
  \begin{itemize}
    \item Lighthouse da falsas mediciones para aplicaciones Next.js.
    \item Lighthouse estima las Web Vitals ejecutando una simulación.
    \item En este caso, usar Vercel Analytics aporta ventajas, como datos reales de los dispositivos de los usuarios.
  \end{itemize}

\end{frame}
\begin{frame}
  \frametitle{Vercel Analytics}
  \framesubtitle{Experiencia de Usuario de la Página Web}
  \centering
  \includegraphics[clip=true,width=\textwidth]{Vercel/VercelAnalytics}\\
\end{frame}


% ==================== CONCLUSIONES ========================
\section{Conclusiones}
\begin{frame}
  \frametitle{Conclusiones}
  Objetivos Alcanzados:
    \begin{itemize}
      \item OBJ1  \smallskip
      \item OBJ2 \bigskip
    \end{itemize}
  Trabajos Futuros:
    \begin{itemize}
      \item Funcionalidades: Ejercicios de acordes, ritmos, progresiones, modo nocturno y varios idiomas. \smallskip
      \item Añadir cobertura de código y test automáticos al flujo de Integración Continua (CI) \smallskip
      \item Considerar Software libre vs Negocio: Plantear alguna forma de monetización, marketing, etc. \bigskip
    \end{itemize}
\end{frame}

\begin{frame}
  \frametitle{Conclusiones}
  Buenas Prácticas:
  \begin{itemize}
    \item Guía de Estilo de Código \smallskip
    \item Principios Clean Code \smallskip
    \item Otras Buenas Prácticas \bigskip
  \end{itemize}
\end{frame}

% ==================== DEMO ========================
\section{Demo}
\begin{frame}
  \frametitle{Demo}
  \framesubtitle{Video de demostración}
\end{frame}

\setbeamercolor{math text}{fg=Crimson}
\setbeamercolor{math text inlined}{parent=math text}
\setbeamercolor{math text displayed}{parent=math text}

% ==================== ÚLTIMA DIAPOSITIVA ========================

\usebackgroundtemplate{\includegraphics[width=\paperwidth]{fondo_blanco.jpg}}

\begin{frame}[plain]

  \begin{center}

    \bigskip  \bigskip

    \begin{Large} \textcolor{crimson}{\textbf{\titulotrabajo}} \end{Large} \bigskip \bigskip

    \begin{footnotesize} \textrm{Trabajo Fin de Grado} \\ \end{footnotesize} \bigskip \bigskip

    \begin{footnotesize} \textbf{\grado} -- \curso \end{footnotesize} \bigskip \bigskip

    \begin{footnotesize} \textsf{Autor:} \textit{\nombreautor} \\ \smallskip
                          \textsf{Tutor:} \textit{\nombretutor} \end{footnotesize} \bigskip \bigskip


    \includegraphics[height=1cm]{logo_ETSII_estrecho.jpg}

  \end{center}

\end{frame}


% ==================== Ejemplos Latex========================
\begin{frame}
  \frametitle{Ejemplo con colores}

  \begin{itemize}
    \item $\mathbf{x}$: Data sample \smallskip
    \item $\mathbf{p}$: Low-dimensional representation of $\mathbf{x}$  \smallskip
    \item $\hat{\mathbf{x}} = \mathbf{V}\mathbf{p}$: Estimates of $\mathbf{x}$ \smallskip
    \item $\hat{x}_{i} = \mathbf{v}_{i}^{\traspuesta}\mathbf{p}$: Estimate of $\mathbf{x}$ for the $i$-th attribute \smallskip
    \begin{itemize}
      \item Through orthogonal projections onto $i$-th axis
    \end{itemize}
  \end{itemize}

\end{frame}

 % === Ejemplo con minipage ===
 \usebackgroundtemplate{\includegraphics[width=\paperwidth]{fondo_blanco.jpg}}
 \begin{frame}
   \frametitle{Ejemplo con minipage}
 
   \begin{minipage}{0.45\textwidth}
    \centering
 
 \begin{itemize}
   \item Teorema de Pitágoras \medskip
   \item Hipotenusa  \medskip
   \item Catetos \medskip
   \item Triángulo rectángulo \medskip
   \begin{itemize}
     \item Incluid espacios verticales \smallskip
     \item Se ha quitado la marca de agua
   \end{itemize}
 \end{itemize}
 
   \end{minipage}
   \hfill
   \begin{minipage}{0.45\textwidth}
    \centering
      \includegraphics[clip=true,width=\textwidth]{triangulos_separados_bb.pdf}\\
   \end{minipage}
 
 \end{frame}
 
 % Restablecer la marca de agua
 \usebackgroundtemplate{\includegraphics[width=\paperwidth]{logoURJC_marca_agua_U.jpg}}
 % === Fin Ejemplo con minipage ===


\end{document}



  %\begin{itemize}
  %  \item Instrument \smallskip
  %  \item InstrumentName \smallskip
  %  \item NotePlayer \smallskip
  %  \item VariantExercise \smallskip
  %  \item Answer \smallskip
  %  \item SelectableAnswer \smallskip
  %  \item SelectableAnswerWithColor \bigskip
  %\end{itemize}

  % \subsection{Hooks y Context}
% \begin{frame}
%   \frametitle{Hooks}
%   \begin{itemize}
%     \item \textbf{useExercise.tsx} \smallskip
%     \item \textbf{useAnswerToggles.tsx} \smallskip
%     \item \textbf{useAnswer.tsx} \smallskip
%     \item \textbf{useAnswerButtons.tsx} \smallskip
%     \item \textbf{useStreak.tsx} \smallskip
%     \item \textbf{usePlayButton.tsx} \smallskip
%     \item \textbf{useScaleDropDown.tsx} \smallskip
%     \item \textbf{usePiano.tsx} \bigskip
%   \end{itemize}
% \end{frame}

% \begin{frame}
%   \frametitle{Context}
%   \begin{itemize}
%     \item \textbf{EarfitContex.tsx}: Se encarga de la lógica relacionada a los instrumentos.\bigskip
%   \end{itemize}
% \end{frame}