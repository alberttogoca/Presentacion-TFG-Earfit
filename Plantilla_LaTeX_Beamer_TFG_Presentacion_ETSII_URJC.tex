
%\documentclass[xcolor=svgnames,handout,aspectratio=169]{beamer} % proyector ancho (aspecto 16x9)
\documentclass[xcolor=svgnames,handout]{beamer} % proyector tradicional (aspecto 4x3)

%%%% usa PDFLATEX !!!!!!!!!!!!!!!!!! %%%%



\mode<presentation>
{
  %% THEME %%
  \usetheme{Frankfurt}
  %\usecolortheme{beaver}



\setbeamertemplate{items}[ball]
\setbeamertemplate{blocks}[rounded][shadow=true]
\setbeamertemplate{navigation symbols}{} %%% To get rid of navigating icons


%\usecolortheme{rose}


\usecolortheme[named=Crimson]{structure}



%\setbeamercolor{palette primary}{bg=lightred,fg=darkred}


\setbeamercolor{palette primary}{bg=rojoURJC,fg=white}
\setbeamercolor{palette secondary}{bg=midred,fg=white}
\setbeamercolor{palette tertiary}{bg=lightred,fg=darkred}


\setbeamercolor*{author in head/foot}{parent=palette tertiary}
\setbeamercolor*{title in head/foot}{parent=palette secondary}
\setbeamercolor*{date in head/foot}{parent=palette primary}

% Color de fórmulas. Se puede usar MediumBlue, Crimson, etc.
% En este documento aparecen en negro
\setbeamercolor{math text}{fg=Black}
\setbeamercolor{math text inlined}{parent=math text}
\setbeamercolor{math text displayed}{parent=math text}

\setbeamercolor{alerted text}{fg=Red}


\defbeamertemplate*{footline}{infolines theme}
{
  \leavevmode%
  \hbox{%
  \begin{beamercolorbox}[wd=.333333\paperwidth,ht=2.25ex,dp=1ex,center]{author in head/foot}%
    \usebeamerfont{author in head/foot}\insertshortinstitute
  \end{beamercolorbox}%
  \begin{beamercolorbox}[wd=.333333\paperwidth,ht=2.25ex,dp=1ex,center]{title in head/foot}%
    \usebeamerfont{title in head/foot}\insertshortauthor
  \end{beamercolorbox}%
  \begin{beamercolorbox}[wd=.333333\paperwidth,ht=2.25ex,dp=1ex,right]{date in head/foot}%
    \usebeamerfont{date in head/foot}\insertshortdate{}\hspace*{2em}
    \insertframenumber{} / \inserttotalframenumber\hspace*{2ex}
  \end{beamercolorbox}}%
  \vskip0pt%
}



  %% FONT %%
  %\usefonttheme{default}
  %\usefonttheme{professionalfonts}
  %\usefonttheme{serif}
  \usefonttheme{structurebold}
  %\usefonttheme{structureitalicserif}
  %\usefonttheme{structuresmallcapsserif}


%  \setbeamercovered{transparent}
%  \setbeamercovered{dynamic}
}


\usepackage[spanish]{babel}

%\usepackage[latin1]{inputenc} % Usad en WinEdt/MikTex
\usepackage[utf8]{inputenc} % Usad en overleaf

\usepackage{graphicx, amsmath}

%\usepackage{paralist}

\usepackage{hyperref}
\usepackage{multimedia}
\usepackage{color}


\usepackage{pgfpages}
%\pgfpagesuselayout{2 on 1}[a4paper,border shrink=5mm]
%\pgfpagesuselayout{4 on 1}[landscape,a4paper,border shrink=5mm]

\definecolor{darkblue}{rgb}{0.15,0.15,0.70}
%\definecolor{darkred}{rgb}{0.80,0.20,0.20}


\definecolor{crimson}{rgb}{0.86,0.08,0.24}
\definecolor{darkred}{rgb}{0.65,0.17,0.17}
\definecolor{midred}{rgb}{0.85,0.40,0.40}
\definecolor{lightred}{rgb}{0.95,0.65,0.65}

\definecolor{rojoURJC}{rgb}{0.796,0,0.09}
\definecolor{pureblue}{rgb}{0,0,1}





%%%%%%%%%%%%%%%%%%%%%%% Definiciones básicas %%%%%%%%%%%%%%%%%%%%%%%

\newcommand{\nombreautor}{Nombre Apellido1 Apellido2}
\newcommand{\nombretutor}{NombreTutor Apellido1 Apellido2}
\newcommand{\titulotrabajo}{Título del Trabajo de Fin de Grado -- Título del Trabajo de Fin de Grado}
\newcommand{\escuela}{Escuela Técnica Superior\\de Ingeniería Informática}
\newcommand{\escuelalargo}{Escuela Técnica Superior de Ingeniería Informática}
\newcommand{\universidad}{Universidad Rey Juan Carlos}
\newcommand{\fecha}{XX/XX/20XX}
\newcommand{\grado}{Grado en XXXXXXXX}
\newcommand{\gradoabr}{TFG - GII} % Para el Grado en Ingeniería Informática
\newcommand{\curso}{Curso 20XX-20XX}
\newcommand{\logoUniversidad}{logoURJC.pdf}
\newcommand{\logoEscuela}{logo_ETSII_estrecho.jpg}

%%%%%%%%%%%%%%%%%%%%%%%%%%%%%%%%%%%%%%%%%%%%%%%%%%%%%%%%%%%%%%%%%%%%


%%%%%%%%%%%%%%%%%%%%%%%%%%%% Comandos definidos por el autor %%%%%%%

\newcommand{\traspuesta}{\mbox{\tiny $\mathsf{T}$}}

%%%%%%%%%%%%%%%%%%%%%%%%%%%%%%%%%%%%%%%%%%%%%%%%%%%%%%%%%%%%%%%%%%%%





\title{\titulotrabajo}
\author[\gradoabr]{\footnotesize \textrm{Trabajo fin de grado}}
\institute[\textsc{\nombreautor}]{\large \textbf{\grado} \\ \bigskip   \bigskip   \begin{footnotesize} \textsf{Autor:} \textit{\nombreautor} \\ \smallskip \textsf{Tutor:} \textit{\nombretutor} \end{footnotesize} \\ \bigskip \bigskip \includegraphics[height=1cm]{\logoEscuela}}
\date[\fecha]{%\vspace{-0.7cm}
}





\begin{document}


\begin{frame}[plain]
\maketitle
\end{frame}








% Opcionales
\usebackgroundtemplate{\includegraphics[width=\paperwidth]{logoURJC_marca_agua_U.jpg}}
\logo{\vspace{-0.2cm} \includegraphics[height=0.5cm,clip=true]{icono_grado_informatica.png}}



\section*{Contenido}

\begin{frame}
  \frametitle{Contenido}

  \tableofcontents

\end{frame}



\section{Objetivos}

\begin{frame}
  \frametitle{Objetivos del TFG}

  \begin{itemize}
    \item Objetivos generales \smallskip
    \begin{itemize}
      \item Obj 1 \smallskip
      \item Obj 2 \bigskip
    \end{itemize}
    \item Objetivos específicos \smallskip
    \begin{itemize}
      \item \alert{Texto alerta} \bigskip
    \end{itemize}
  \end{itemize}

\end{frame}



\subsection{Sub sec 1}

\begin{frame}
  \frametitle{Objetivos del TFG}
  \framesubtitle{Usad subsecciones}

  \begin{itemize}
    \item Objetivos generales \smallskip
    \begin{itemize}
      \item Obj 1 \smallskip
      \item Obj 2 \bigskip
    \end{itemize}
    \item Objetivos específicos \smallskip
    \begin{itemize}
      \item \alert{Texto alerta} \bigskip
    \end{itemize}
  \end{itemize}

\end{frame}


\begin{frame}
  \frametitle{Objetivos del TFG}

  \begin{itemize}
    \item Objetivos generales \smallskip
    \begin{itemize}
      \item Obj 1 \smallskip
      \item Obj 2 \bigskip
    \end{itemize}
    \item Objetivos específicos \smallskip
    \begin{itemize}
      \item \alert{Texto alerta} \bigskip
    \end{itemize}
  \end{itemize}

\end{frame}

\subsection{Sub sec 2}

\begin{frame}
  \frametitle{Objetivos del TFG}

  \begin{itemize}
    \item Objetivos generales \smallskip
    \begin{itemize}
      \item Obj 1 \smallskip
      \item Obj 2 \bigskip
    \end{itemize}
    \item Objetivos específicos \smallskip
    \begin{itemize}
      \item \alert{Texto alerta} \bigskip
    \end{itemize}
  \end{itemize}

\end{frame}


\begin{frame}
  \frametitle{Objetivos del TFG}

  \begin{itemize}
    \item Objetivos generales \smallskip
    \begin{itemize}
      \item Obj 1 \smallskip
      \item Obj 2 \bigskip
    \end{itemize}
    \item Objetivos específicos \smallskip
    \begin{itemize}
      \item \alert{Texto alerta} \bigskip
    \end{itemize}
  \end{itemize}

\end{frame}



\section{Desarrollo}

\begin{frame}
  \frametitle{Enumeración}

  \begin{enumerate}
    \item Punto 1 \bigskip
        \pause
    \item Punto 2 \bigskip
        \pause
    \item Punto 3 \bigskip
        \pause
    \item Punto 4 \bigskip
        \pause
    \item Punto 5 \bigskip
  \end{enumerate}

\end{frame}



\begin{frame}
  \frametitle{Fórmulas}

Problema de optimización:
\[
\begin{array}{cl}
  \displaystyle \begin{array}{c}\mathrm{minimizar} \\ \mathbf{t} \in \mathbb{R}^{n}, \  \mathbf{p} \in \mathbb{R}^{m} \end{array} & \hspace{-0.2cm} \begin{array}{c} \mathbf{1}^{\traspuesta}\mathbf{t} \\ \mbox{} \end{array}  \\
  & \vspace{-0.4cm} \\ % línea (fila) en blanco, pero la hacemos estrecha con el comando vspace
  \mbox{sujeto a} & -\mathbf{t} \preceq  \mathbf{V}\mathbf{p} - \mathbf{x}  \preceq  \mathbf{t},\\
 \end{array}
\]

\end{frame}



\section{Resultados}


\usebackgroundtemplate{\includegraphics[width=\paperwidth]{fondo_blanco.jpg}}
\begin{frame}
  \frametitle{Ejemplo con minipage}

  \begin{minipage}{0.45\textwidth}
   \centering

\begin{itemize}
  \item Teorema de Pitágoras \medskip
  \item Hipotenusa  \medskip
  \item Catetos \medskip
  \item Triángulo rectángulo \medskip
  \begin{itemize}
    \item Incluid espacios verticales \smallskip
    \item Se ha quitado la marca de agua
  \end{itemize}
\end{itemize}

  \end{minipage}
  \hfill
  \begin{minipage}{0.45\textwidth}
   \centering
     \includegraphics[clip=true,width=\textwidth]{triangulos_separados_bb.pdf}\\
  \end{minipage}

\end{frame}

% Restablecer la marca de agua
\usebackgroundtemplate{\includegraphics[width=\paperwidth]{logoURJC_marca_agua_U.jpg}}




\begin{frame}
  \frametitle{Constrained norm minimization problem}
  \framesubtitle{Exact estimates for one data variable}
%  \thispagestyle{empty}


\[
  \begin{array}{cl}
  \displaystyle \begin{array}{c}\mathrm{minimize} \\ \mathbf{p} \in \mathbb{R}^{m} \end{array} & \hspace{-0.2cm} \begin{array}{c} \| \hat{\mathbf{x}} - \mathbf{x} \|, \\ \mbox{} \end{array}  \\[15pt]
   \mbox{subject to} & \mathbf{v}_{i}^{\traspuesta}\mathbf{p} = x_{i}, \\
  \end{array}
\]

\begin{itemize}
  \item $\mathbf{x}$: Data sample \smallskip
  \item $\mathbf{p}$: Low-dimensional representation of $\mathbf{x}$  \smallskip
  \item $\hat{\mathbf{x}} = \mathbf{V}\mathbf{p}$: Estimates of $\mathbf{x}$ \smallskip
  \item $\hat{x}_{i} = \mathbf{v}_{i}^{\traspuesta}\mathbf{p}$: Estimate of $\mathbf{x}$ for the $i$-th attribute \smallskip
  \begin{itemize}
    \item Through orthogonal projections onto $i$-th axis
  \end{itemize}
\end{itemize}


\end{frame}



\setbeamercolor{math text}{fg=pureblue}
\setbeamercolor{math text inlined}{parent=math text}
\setbeamercolor{math text displayed}{parent=math text}

\begin{frame}
  \frametitle{Constrained norm minimization problem}
  \framesubtitle{Exact estimates for one data variable}
%  \thispagestyle{empty}


\[
  \begin{array}{cl}
  \displaystyle \begin{array}{c}\mathrm{minimize} \\ \mathbf{p} \in \mathbb{R}^{m} \end{array} & \hspace{-0.2cm} \begin{array}{c} \| \hat{\mathbf{x}} - \mathbf{x} \|, \\ \mbox{} \end{array}  \\[15pt]
   \mbox{subject to} & \mathbf{v}_{i}^{\traspuesta}\mathbf{p} = x_{i}, \\
  \end{array}
\]

\begin{itemize}
  \item $\mathbf{x}$: Data sample \smallskip
  \item $\mathbf{p}$: Low-dimensional representation of $\mathbf{x}$  \smallskip
  \item $\hat{\mathbf{x}} = \mathbf{V}\mathbf{p}$: Estimates of $\mathbf{x}$ \smallskip
  \item $\hat{x}_{i} = \mathbf{v}_{i}^{\traspuesta}\mathbf{p}$: Estimate of $\mathbf{x}$ for the $i$-th attribute \smallskip
  \begin{itemize}
    \item Through orthogonal projections onto $i$-th axis
  \end{itemize}
\end{itemize}


\end{frame}

\setbeamercolor{math text}{fg=Crimson}
\setbeamercolor{math text inlined}{parent=math text}
\setbeamercolor{math text displayed}{parent=math text}


\begin{frame}
  \frametitle{Constrained norm minimization problem}
  \framesubtitle{Exact estimates for one data variable}
%  \thispagestyle{empty}


\[
  \begin{array}{cl}
  \displaystyle \begin{array}{c}\mathrm{minimize} \\ \mathbf{p} \in \mathbb{R}^{m} \end{array} & \hspace{-0.2cm} \begin{array}{c} \| \hat{\mathbf{x}} - \mathbf{x} \|, \\ \mbox{} \end{array}  \\[15pt]
   \mbox{subject to} & \mathbf{v}_{i}^{\traspuesta}\mathbf{p} = x_{i}, \\
  \end{array}
\]

\begin{itemize}
  \item $\mathbf{x}$: Data sample \smallskip
  \item $\mathbf{p}$: Low-dimensional representation of $\mathbf{x}$  \smallskip
  \item $\hat{\mathbf{x}} = \mathbf{V}\mathbf{p}$: Estimates of $\mathbf{x}$ \smallskip
  \item $\hat{x}_{i} = \mathbf{v}_{i}^{\traspuesta}\mathbf{p}$: Estimate of $\mathbf{x}$ for the $i$-th attribute \smallskip
  \begin{itemize}
    \item Through orthogonal projections onto $i$-th axis
  \end{itemize}
\end{itemize}


\end{frame}




\usebackgroundtemplate{\includegraphics[width=\paperwidth]{fondo_blanco.jpg}}

\begin{frame}[plain]

\begin{center}

  \bigskip  \bigskip

  \begin{Large} \textcolor{crimson}{\textbf{\titulotrabajo}} \end{Large} \bigskip \bigskip

  \begin{footnotesize} \textrm{Trabajo Fin de Grado} \\ \end{footnotesize} \bigskip \bigskip

  \begin{footnotesize} \textbf{\grado} -- \curso \end{footnotesize} \bigskip \bigskip

  \begin{footnotesize} \textsf{Autor:} \textit{\nombreautor} \\ \smallskip
                        \textsf{Tutor:} \textit{\nombretutor} \end{footnotesize} \bigskip \bigskip


  \includegraphics[height=1cm]{logo_ETSII_estrecho.jpg}

\end{center}

\end{frame}



\end{document}
